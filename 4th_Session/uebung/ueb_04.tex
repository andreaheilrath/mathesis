\input setup_ueb

\usepackage[T1]{fontenc}


\begin{document}

\section*{Zusatzaufgaben: Objekte, Klassen}

Jede dieser Aufgaben ist soviel Wert wie ein ganzes Hausaufgabenblatt.
Die Aufgaben könnt ihr im Laufe des Semesters lösen, indem ihr eine
entsprechende Datenstruktur baut. Die Punkte können als Zusatzpunkte
fehlende Punkte bei den drei Hausaufgabenblättern ausgleichen.
%Jede dieser Aufgaben ist 10 Punkte wert (die mit fehlenden Punkten der drei Blätter
%verrechnet werden können.) Bei etlichen Aufgaben müssen Sie die Details recherchieren.

\begin{enumerate}[1.]
\item \textbf{Eine Klasse für Daten}

Definieren Sie eine Klasse \texttt{Data}, die als Datencontainer dienen
soll. Sie soll eine Methode \texttt{titles()} haben, die eine Liste
der Bezeichnungen der bisher gespeicherten Daten enthält, sowie eine Methode
\texttt{set\_data}, die einen Namen und ein Datenobjekt übergeben bekommt
und dieses zu dem Namen in einem Wörterbuch speichert, sowie eine
Methode \texttt{get\_data}, die einen Namen übergeben bekommt und
das Datenobjekt zurückgibt.  Weiterhin soll es noch eine
Methode \texttt{set\_properties} und eine Methode \texttt{get\_properties}
geben, die zu einem Namen ein Dictionary von Eigenschaften speichert.
Außerdem soll die Klasse Methoden \texttt{pickle} und \texttt{unpickle}
haben, die einen Dateinamen übergeben bekommen und das Objekt 
mit Hilfe des Moduls \texttt{pickle} speichern bzw. lesen.

Wenn Sie eine besondere Art von Daten speichern wollen, können Sie
 von dieser Klasse eine andere ableiten, z.B. \texttt{EEGData},
und dort noch spezielle Methoden zum Lesen von Daten in gewissen
Formaten und zum Auswerten, Anzeigen, etc. hinzufügen.  Dabei ist
es wichtig, sich die Eigenschaften der entsprechenden Daten
zu überlegen (z.B. bei einer Tabelle die Namen der Spalten oder
der Zeilen, Einheiten, was weiß ich), um diese Eigenschaften in
den Auswertungsfunktionen benutzen zu können.

\item \textbf{Eine Klasse für Kompositionen}

Definieren sie ein Klasse \texttt{Komposition}, 
die eine Komposition repräsentiert.
Die Aufgabe ist offen gehalten. Die Klasse kann als Attribute
etwa \texttt{akkordschema} und \texttt{stimmen} enthalten.
Diese Attributen wären dann Listen von Objekten der Klasse
\texttt{Akkord}, bzw. \texttt{Stimme}.  Wie Sie 
weitermachen, können Sie überlegen. Stimmen wiederum
bestünden aus einer Abfolge von Tönen. Die Klasse
\texttt{Ton} sollte mindestens die Attribute 
\texttt{dauer, notenwert, lautstaerke} haben.  Diese Objekte können
auch Abspielmethoden haben, die sie hörbar machen.

\item \textbf{Eine Klasse für Himmelskörper}

Definieren Sie eine Klasse \texttt{Body}, die als Attribute
die Masse, den Radius, den Namen des Himmelskörpers enthält.
Der Konstruktor sollte diese Parameter übergeben bekommen.
Weitere mögliche Attribute: Bahnparameter, um die Keplerschen
Bahnen zu berechnen, oder die Position und Geschwindigkeit 
zu einem gewissen Zeitpunkt, was ebenfalls die Berechnung der Bahnen
erlaubt.   

Definieren Sie eine Klasse \texttt{Viewer}, die
in ihrem Konstruktor eine Liste von Himmelskörpern
bekommt und Methoden \texttt{visualize}
und \texttt{update}, die diese Himmelskörper 3d visualisieren,
bzw. die Positionen aktualisieren.
Da in galaktischen Maßstäben die Himmelskörper oft furchtbar
klein sind, sollte die Methode \texttt{visualize} einen 
optionalen Paramtere \texttt{scale} haben, mit dem
man die Himmelskörper vergrößern kann. Am besten eignet
sich dafür das Moduld \texttt{vpython} (es heißt auch
'visual python' oder 'visual'.)
\item \textbf{Game of Life}

Programmieren Sie das Game of Life. Dafür ist es
angemessen, eine Klasse \texttt{Welt}  zu definieren, die den Zustand 
der Welt in einem Attribut speichert und eine Methode \texttt{schritt} hat, 
die einen Entwicklungssschritt beschreibt. Weiterhin kann die
Klasse eine Methode \texttt{anzeigen} haben, die mit Hilfe von
matplotlib den Zustand anzeigt.

\end{enumerate}
\end{document}