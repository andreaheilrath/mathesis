\input setup_ueb
\begin{document}
\section*{Das 0. Aufgabenblatt...}

\begin{enumerate}[1.]
\item Installieren Sie sich Python 3 auf Ihrem Computer. Verwenden Sie am
besten die Distribution Anaconda, Hinweise stehen auch auf der ISIS-Seite.
\item Installieren Sie auch den Editor \texttt{geany}. (Wenn Sie schon programmieren und dafür einen anderen Editor verwenden, können Sie natürlich auch diesen benutzen.)
\item Öffnen Sie ein so genanntes \textbf{Terminal (Linux/MacOS)} oder 
\textbf{Anaconda Prompt (Windows)}. \\
Es handelt sich um einen Interpreter, aber nicht für Python, sondern für
die Kommandosprache des Betriebssystems. Befehle werden normalerweise in diesem 
Interpreter durch die Eingabetaste abgeschlossen. Lassen Sie
sich mit \texttt{ls} (Linux/Mac), bzw. \texttt{dir} (Windows) den
Inhalt des Verzeichnisses anzeigen. Wechseln Sie mit \texttt{cd $<$Verzeichnisname$>$} in ein Unterverzeichnis und mit \texttt{cd ..} zurück.
\item Starten Sie den interaktiven Python-Interpreter (im Terminal \texttt{python}+ Eingabetaste) und berechnen Sie \texttt{2**1000}.
\item Geben Sie im interaktiven Python-Interpreter\\
\texttt{import math}\\
ein und berechnen Sie anschließend:\\
\texttt{math.cos(math.pi)}\\
Fällt Ihnen eine mathematische Funktion ein, mit deren Hilfe Sie 
die Anzahl der Stellen von \texttt{2**1000} bestimmen können? Finden
Sie die Funktionen im Modul \texttt{math} heraus, indem Sie\\
\texttt{help(math)}
eingeben.
\item Schreiben Sie mit dem Editor \texttt{geany} eine Datei \texttt{hello.py},
die die zwei Zeilen \texttt{print('Hello World!')} und \texttt{wert = 3**333}
enthält.  Führen Sie dieses Programm auf zwei verschiedene Weisen aus:\\
\texttt{python hello.py}
und interaktiv:
\texttt{python -i hello.py}
Geben Sie im interaktiven Fall anschließend 'wert' ein, um sich den Wert
der Variablen 'wert' anzeigen zu lassen.
\item Schreiben Sie mit dem Editor  eine Datei \texttt{hello2.py} mit dem folgenden Inhalt:
\begin{lstlisting}
for zaehler in range(1,100):
    print(zaehler, 'Hallo Welt!')
\end{lstlisting}
Führen Sie dieses Program wie oben durch \texttt{python hello2.py} aus.
Ändern Sie die Zahlen in dem Programm und führen Sie es noch einmal aus.
(Was da eigentlich geschieht, wird später erklärt.)
\newpage
\item Starten Sie ein Jupyter Notebook, mit dem Befehl\\
\texttt{jupyter notebook}\\
Geben Sie einen Rechenausdruck in einer Zelle ein und
rufen Sie durch \texttt{Umschalttaste+Eingabetaste} (Shift+Enter) 
den Interpreter auf.\\
Stellen Sie den Zellentyp von \texttt{Code} zu \texttt{Markdown} um 
und geben Sie ein \\
\verb|# Ein Notebook|\\
\verb|Erste Versuche|\\
und evaluieren Sie die Zelle. Die Zelle wird in einem gewissen
Format gesetzt, aber nicht an den Python-Interpreter übergeben.
\item Rufen Sie einen Python-Interpreter auf und arbeiten Sie die Einführung
\url{https://docs.python.org/3/tutorial/introduction.html} durch.
Probieren Sie die Code-Beispiele jeweils aus und spielen Sie damit herum.
Falls Sie noch keinen Interpreter haben, können Sie auch einen  Online-Python-Interpreter
verwenden, etwa \url{https://www.onlinegdb.com/online_python_compiler}.
\end{enumerate}
\end{document}
